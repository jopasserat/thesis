This thesis is concerned with the problem of preventing
regressions in (legacy) production systems such as those of our
industrial partner Michelin, one of the three largest tire
manufacturers in the world, by means of Model-based Testing. A
production system is defined as a set of production machines
controlled by a software, in a factory.  Despite the large body
of work within the field of Model-based Testing, a common issue
remains the writing of models describing either the system under
test or its specification. It is a tedious task that is even more
complicated when the systems are referenced as "legacy",
\emph{i.e.} out of date, and often not (or poorly) documented. A
second point to take into account is that production systems
often run continuously and should not be disrupted, which makes
most of the existing classical testing techniques unusable.

We present an approach to infer exact models from traces,
\emph{i.e.} sequences of events observed in a production
environment, to address the first issue. We leverage the data
exchanged among the devices and software in a black-box
perspective to construct behavioral models using different
techniques such as expert systems, model inference, and machine
learning. It results in large yet partial models gathering the
behaviors recorded on a system under analysis. We introduce a
context-specific reduction algorithm to reduce such models in
order to make them more usable while preserving trace equivalence
between the original inferred models and the reduced ones. These
models can serve different purposes, \emph{e.g.}, generating
documentation, data mining, but also testing.

To address the problem of testing production systems without
disturbing them, this thesis introduces two passive Model-based
Testing techniques. The offline passive testing technique allows
to detect differences between two production systems, and the
online technique enables on-the-fly testing of production
systems. Both techniques leverage the inferred models, and rely
on two implementation relations: the existing trace preorder
relation, and a weaker implementation proposed to overcome the
partialness of the inferred models.

Overall, the thesis presents \emph{Autofunk}, a modular framework
for model inference and testing of production systems, gathering
the previous notions. Its Java implementation has been applied to
different applications and production systems, and this thesis
gives results from different case studies.
