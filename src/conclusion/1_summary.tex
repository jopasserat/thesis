\section{Summary of achievements}

This thesis has proposed a novel approach to infer small models
of software systems in order to perform Model-based Testing of
production systems. In addition, it has provided a conformance
testing technique that leverages such models, along with two
different implementation relations. The original aims and
objectives of the thesis were as follows:

\begin{itemize}
    \item To infer partial yet exact models of production systems
        in a fast and efficient manner, based on the data
        exchanged in a (production) environment;

    \item To generate test cases based on these models in order
        to perform conformance testing of production systems. The
        main idea was to detect regressions across similar
        production systems (e.g., a software update or a hardware
        upgrade).
\end{itemize}

Chapters \ref{sec:modelinf:webapps} and
\ref{sec:modelinf:prodsystems} addressed the first of the
objectives by proposing two approaches combining model inference
and expert systems to infer exact models for web applications and
industrial systems, wrapped into the \textit{Autofunk} framework.
The expert system is composed of rules used either to filter the
trace set to remove the undesired ones, to infer Input Output
Symbolic Transition Systems (IOSTS) or to build more abstract
IOSTSs. In Chapter \ref{sec:modelinf:prodsystems}, the state
merging is replaced with a context-specific state reduction based
on an event sequence abstraction. This state reduction can be
seen as the kTail algorithm \cite{5009015} where $k$ is as high
as possible.  This state reduction first ensures that the
resulting models do not over-approximate the system under
analysis but it is also very context-specific, and cannot be
generalized. We also showed that our approach is scalable: it can
take thousands and thousands of traces and can still build models
quickly thanks to our specific state merging process.

The second objective was achieved by enhancing \textit{Autofunk}
with a passive testing technique as presented in Chapter
\ref{sec:testing}. Given a large set of production events,
\textit{Autofunk} reuses the inferred models as specification to
perform offline passive testing, using a second set of traces
recorded on a system under test, and two implementation relations
to determine what has changed between the two systems. This is
particularly useful for our industrial partner Michelin since
potential regressions can be detected while deploying changes in
production.

The next section introduces some perspectives for future work on
model inference, while Section \ref{sec:conclusion:testing} is
dedicated to future work on the testing part of our work. Section
\ref{sec:conclusion:final-thoughts} closes this thesis.
