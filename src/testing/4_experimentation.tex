\section{Implementation and experimentation}
\label{sec:testing:offline:impl-exp}

We conducted several experiments with real sets of production
events, recorded in one of Michelin's factories at different
periods of time. The results given in this section are focused on
our \textbf{offline} passive testing technique. We executed our
implementation on a Linux (Debian) machine with 12 Intel(R)
Xeon(R) CPU X5660 @ 2.8GHz and 64GB RAM.

We present, in Figure \ref{fig:testing:offline:results}, the results of several
experiments on the same production system with different trace
sets, recorded at different periods of time. The first column
shows the experiment number, columns 2 and 3 respectively give
the sizes of the trace sets of the system under analysis
$\mathit{Sua}$ and of the system under test $\mathit{Sut}$. The
two next columns show the percentage of pass traces w.r.t the
relations $\leq_{ct}$ and $\leq_{mct}$. The last column indicates
the execution time for the testing phase.

\begin{table}[h]
\begin{center}
\begin{tabular}{| c | c | c | c | c | c |}
\hline
Exp. & $Card(Traces({Sua}))$ & $Card(CTraces({Sut}))$ & Pass$\leq_{ct}$ & Pass$\leq_{mct}$ & Time\\
\hline
\hline
$1$ & 2,075 & 2,075 & 100\% & 100\% & 1 \\
\hline
$2$ & 53,996 & 2,075 & 3\% & 30\% & 4\\
\hline
$3$ & 53,996 & 25,047 & 98\% & 98\% & 10\\
\hline
\end{tabular}
\end{center}

    \caption{This table shows the results of our offline passive
    testing method based on a same specification.}
    \label{fig:testing:offline:results}
\end{table}

In Experiment $1$, we decided to use the same production events
for both inferring models, i.e. specifications, and testing. This
experiment shows that our implementation behaves correctly when
trace sets are similar, i.e. when behaviours of both
$\mathit{Sua}$ and $\mathit{Sut}$ are equivalent.

Experiment $2$ has been run with traces of $\mathit{Sut}$ that
are older than those of $\mathit{Sua}$, which is unusual as the
de facto usage of our framework is to build specifications from a
production system $\mathit{Sua}$, and to take a newer or updated
system as $\mathit{Sut}$.  Here, only 30\% of the traces of
$\mathit{Sut}$ are pass traces w.r.t.  the second implementation
relation (same sequence of symbols with different values). There
are two explanations: the system has been updated between the two
periods of record (4 months), and production campaigns, i.e.
grouping of planned orders and process orders to produce a
certain amount of products over a certain period of time, were
different (revealed by \textit{Autofunk}, indicating that values
for some key parameters were unknown).

Finally, experiment $3$ shows good results as the specification
models are rich enough, i.e.  built from a larger set of traces
(10 days) than the one collected on $\mathit{Sut}$. Such an
experiment is a typical usage of our framework at Michelin.  The
traces of $\mathit{Sut}$ have been collected for 5 days, and it
took only 10 minutes to check conformance. While 98\% of the
traces are pass traces, the remaining 2\% are new behaviours that
never occured before. Such information is essential for Michelin
engineers to determine the root causes. Even though 2\% may
represent a large set to analyse, \textit{Autofunk} improves
their work by highlighting the traces to focus on.  Such subset
may contain false positives depending on the richness of the
models, but using large sets of traces to build the models
reduces the number of false positives.
