\section{Conclusion}
\label{sec:testing:conclusion}

In this chapter, we present a fast passive testing framework
combining different fields such as model inference, expert
systems, and machine learning. Such a framework is an extension
of previous works based on model inference, whose name is
Autofunk.

Given a large set of production events, our framework infers
exact models whose traces are included in the initial trace set
of a system under analysis. Such models are then reused as
specifications to perform: (i) offline passive testing using a
second set of traces recorded on a system under test, (ii) online
passive testing by taking new traces of a system under test
on-the-fly.
Using two implementation relations, Autofunk is able to determine
what has changed between the two systems. This is particularly
useful for our industrial partner Michelin since potential
regressions can be detected while deploying changes in
production. Initial results on the offline merhod are
encouraging, and Michelin engineers see a real potential in this
framework.

An interesting improvement to this framework, directly related to
our industrial partner Michelin, would be to be able to focus on
specific locations of a workshop, rather than on the whole
workshop, since some parts are more critical than others. This
should bring significant improvements to the end users of
Autofunk. Another direction that will be discussed in the next
chapter is to enable \textit{active} testing.
