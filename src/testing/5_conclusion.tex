\section{Conclusion}
\label{sec:testing:conclusion}

In this chapter, we present a fast passive testing framework
combining different fields such as model inference, expert
systems, and machine learning. Such a framework is an extension
of previous works based on model inference, whose name is
\textit{Autofunk}.

Given a large set of production events, our framework infers
exact models whose traces are included in the initial trace set
of a system under analysis. Such models are then reused as
specifications to perform: (i) offline passive testing using a
second set of traces recorded on a system under test, (ii) online
passive testing by taking new traces of a system under test
on-the-fly. Using two implementation relations, \textit{Autofunk}
is able to determine what has changed between the two systems.
This is particularly useful for our industrial partner Michelin
since potential regressions can be detected while deploying
changes in production. Initial results on the offline method are
encouraging, and Michelin engineers see a real potential in this
framework.

We know that 2\% of a large trace set (as mentioned in the
results) can represent many traces, which can still be difficult
to analyze. In a manufacturing context, this is valuable although
we would like to reduce such false negatives. In the next
chapter, we give our thoughts on how to improve \textit{Autofunk}
as well as perspectives to leverage our work.
