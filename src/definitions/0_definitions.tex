% !TEX root = ../../thesis.tex
%
\chapter{Definitions and notations}
\label{sec:definitions}

%%%%%%%%%%%%%%%%%%%%%%%%%%%%%%%%%%%%%%%%%%%%%%%%%%%%%%%%%%%%%%%%%

\section{Symbolic Transition Systems)}
\label{sec:definitions:sts}

The Symbolic Transition System (STS) is known as a very general
and powerful model for describing several aspects of event-based
systems. The use of symbolic variables helps describe infinite
state machines in a finite manner. This potentially infinite
behaviour is represented by the semantics of a STS, given in
terms of Labelled Transition System (LTS). STS operations and
transformations are often given with inference rules.

We briefly give some definitions related to the STS model below,
but we refer to \cite{FTW05} for a more detailed description.

\begin{definition}[Variable assignment]
We assume that there exist a domain of values denoted $D$ and a
variable set $X$ taking values in $D$. The assignment of
variables in $Y \subseteq X$ to elements of $D$ is denoted with a
mapping  $\alpha: Y \rightarrow D$.

We denote $D_Y$ the assignment set over $Y$. We also denote
$id_Y$ the identity assignment over $Y$, and $v_\emptyset$ the
empty assignment.
\end{definition}

\begin{definition}[STS]
    A Symbolic Transition System (STS) is a tuple
    $<L,l_0,V,V_0,I,\Lambda,\rightarrow>$, where:

	\begin{itemize}
        \item STSs do not have states but locations and $L$ is
        the finite location set, with $l_0$ being the initial
        one,

        \item $V$ is the finite set of internal variables, while
        $I$ is the finite set of parameters. The internal
        variables are initialised with the condition $V_0$ on
        $V$,

        \item $\Lambda$ is the finite set of symbolic events
        $a(p)$, with $p=(p_1,...,p_k)$ a finite set of parameters
        in $I^k (k \in \N)$,

        \item $\rightarrow$ is the finite transition set. A
        transition $(l_i,l_j,a(p),G,A)$, from the location $l_i
        \in L$ to $l_j \in L$, also denoted $l_i
        \xrightarrow{a(p),G,A} l_j$ is labelled by:

		\begin{itemize}
            \item an action $a(p) \in \Lambda$,

            \item $G$ is a guard over $(p \cup V \cup T(p \cup
            V))$ which restricts the firing of the transition.
            $T(p \cup V)$ are boolean terms, a.k.a. predicates
            over $p \cup V$,

            \item internal variables are updated with the
            assignment function $A$ of the form $(x:=A_x)_{x \in
            V}$, $A_x$ is an expression over $V\cup p \cup
            T(p\cup V)$.
		\end{itemize}
	\end{itemize}

	\label{def:sts}
\end{definition}

For readability purpose, we also use the generalised transition
relation $\Rightarrow$ to represent STS paths:\\ $l
\xRightarrow{(a_1,G_1,A_1) \dots (a_n,G_n,A_n)} l' =_{def}
\exists l_0 \dots l_n, l=l_0 \xrightarrow{a_1,G_1,A_1} l_1 \dots
l_{n-1}\xrightarrow{a_n,G_n,A_n}l_n=l'$.

A STS is associated with a Labelled Transition System (LTS) to
formulate its semantics. Intuitively, a LTS semantics corresponds
to a valued state machine, without symbolic variables, which is
often infinite: the LTS states are labelled by internal variable
assignments while transitions are labelled by actions combined
with parameter assignments.

\begin{definition}[LTS semantics]
    The semantics of a STS $\EuScript{S}=<L,l_0,$ $V,$ $V_0,$
    $I,\Lambda,\rightarrow>$ is the LTS
    $||\EuScript{S}||=<Q,q_0,\sum,\rightarrow>$ where:

	\begin{itemize}

		\item $Q=L \times D_V$ is the set of states,

        \item $q_0=(l_0,V_0)$ is the initial state,

		\item $\sum=\{(a(p),\alpha)  \mid  a(p)\in\Lambda, \alpha \in
		D_p\}$ is the set of valued events,

        \item $\rightarrow$ is the transition relation $Q \times
        \Sigma \times Q$ deduced by the following rule:\\
	\end{itemize}
	\begin{center}
		\fbox{
			\begin{minipage}{0.6\textwidth}
				\begin{tabular}{l}
					$\frac{l_1 \xrightarrow{a(p),G,A}l_2,\alpha \in D_p, v \in D_V, v'
						\in D_V, v \cup \alpha \models G, v'=A(v \cup \alpha))}{(l_1,v)
						\xrightarrow{a(p),\alpha} (l_2,v') }$
				\end{tabular}
				\newline
			\end{minipage}
		}
	\end{center}

	\label{def:semantics}
\end{definition}

This rule can be read as follows: for a STS transition $l_1
\xrightarrow{a(p),G,A}l_2$, we obtain a LTS transition $(l_1,v)$
$\xrightarrow{a(p),\alpha} (l_2,v')$ with $v$ a variable
assignment over the internal variable set, if there exists an
assignment $\alpha$ such that the guard $G$ evaluates to true
with $v \cup \alpha$. Once the transition is executed, the
internal variables are assigned with $v'$ derived from the
assignment $A(v \cup \alpha)$.

Finally, runs and traces, which represent executions and event
sequences, can also be derived from LTS semantics:

\begin{definition}[Runs and traces]
    Given a STS $\EuScript{S}=$ $<L,l_0,V,V_0,I,\Lambda,
	\rightarrow>$, interpreted by its LTS semantics
	$||\EuScript{S}||=<Q,q_0,\sum,\rightarrow>$, a run $q_0
	\alpha_0 \dots \alpha_{n-1} q_n$ is an alternate sequence of states
    and valued actions. $Run(\EuScript{S})=Run(||\EuScript{S}||)$ is
	the set of runs found in $||\EuScript{S}||$.

    It follows that a trace of a run $r$ is defined as the projection
    $proj_{\sum}(r)$ on the actions.

	\label{def:Runs and traces}
\end{definition}

%%%%%%%%%%%%%%%%%%%%%%%%%%%%%%%%%%%%%%%%%%%%%%%%%%%%%%%%%%%%%%%%%

\section{Input/Output Symbolic Transition Systems}
\label{sec:definitions:iosts}

We consider the input/output Symbolic Transition System (IOSTS)
formalism \cite{FTW05} for describing the functional behaviour of
systems or applications. An IOSTS is a kind of automata model
which is extended with two sets of variables, internal variable
to store data, and parameters to enrich the actions. Transitions
carry actions, guards, and assignments over variables. The action
set can be divided into two subsets: one containing inputs
beginning by $?$ to express actions expected by the system, and
another containing outputs beginning by $!$ to express actions
produced by the system. An IOSTS does not have states but
locations.

\begin{definition}[IOSTS]
An IOSTS $\EuScript{S}$ is a tuple $<
L,l0,V,V0,I,\Lambda,\rightarrow>$, where:

\begin{itemize}
\item $L$ is the finite set of locations, $l0$ the initial
location,

\item $V$ is the finite set of internal variables, $I$ is the
finite set of parameters. We denote $D_v$ the domain in which a
variable $v$ takes values. The assignment of values of a set of
variables $Y \subseteq V \cup I$ is denoted by valuations where a
valuation is a function $v: Y \rightarrow D$. $v_\emptyset$
denotes the empty valuation. $D_Y$ stands for the valuation set
over the variable set $Y$. The internal variables are initialised
with the assignment $V0$ on $V$, which is assumed to be unique,

\item $\Lambda$ is a finite set of symbolic actions $a(p)$, with
$p = (p_1,\dots,p_k)$ a finite list of parameters in $I^k(k \in
\mathbb{N})$. $p$ is assumed unique. $\Lambda= \Lambda^I  \cup
\Lambda^O \cup \{!\delta \}$: $\Lambda^I$ represents the set of
input actions, $\Lambda^O$ the set of output actions, and
$\delta$ the quiescence,

\item $\rightarrow$ is the finite transition set. A transition
$(l_i,l_j,a(p),G,A)$, from the location $l_i \in L$ to $l_j \in
L$, denoted $l_i \xrightarrow{a(p),G,A} l_j$ is labelled by: an
action $a(p) \in \Lambda$, a guard  $G$ over $(p \cup V \cup T(p
\cup V))$ which restricts the firing of the transition. $T(p \cup
V)$ is a set of functions that return boolean values only (a.k.a.
predicates) over $p \cup V$, an assignment function $A$ which
updates internal variables. $A$ is on of the form $(x:=A_x)_{x\in
V}$, where $A_x$ is an expression over $V \cup p \cup T(p \cup
V)$.
\end{itemize}

\end{definition}

An IOSTS is also associated with an IOLTS (Input/Output Labelled
Transition System) to formulate its semantics. Intuitively, IOLTS
semantics correspond to valued automata without symbolic variable,
which are often infinite: IOLTS states are labelled by internal
variable valuations while transitions are labelled by actions and
parameter valuations. The semantics of an IOSTS
$\EuScript{S}=<L,l_0,V,V_0,I,\Lambda,\rightarrow>$ is the IOLTS
$\llbracket \EuScript{S}\rrbracket = <Q,q_0,\Sigma,\rightarrow>$
composed of valued states in $Q = L \times D_V$, $q_0=(l_0,V_0)$
is the initial one, $\Sigma$ is the set of valued symbols and
$\rightarrow$ is the transition relation. The IOLTS semantics
definition can be found in \cite{FTW05}. In short, for an
IOSTS transition $l_1 \xrightarrow{a(p),G,A}l_2$, we obtain an
IOLTS transition $(l_1,v)$ $\xrightarrow{a(p),\theta} (l_2,v')$
with $v$ a set of valuations over the internal variable set, if
there exists a parameter valuation set $\theta$ such that the
guard $G$ evaluates to true with $v \cup \theta$. Once the
transition is executed, the internal variables are assigned with
$v'$ derived from the assignment $A(v \cup \theta)$.
